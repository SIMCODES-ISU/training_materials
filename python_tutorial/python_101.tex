\documentclass[11pt]{beamer}
\usetheme{Boadilla}
\usepackage[utf8]{inputenc}
\usepackage{amsmath,amsfonts,amssymb}
\usepackage{lmodern}
\usepackage{listings}
\usepackage[absolute,overlay]{textpos}
\usepackage[export]{adjustbox}

\usepackage{epigraph} %For quote 
\usepackage{hyperref} %Colors set below to use Ames/ISU colors

%%%%%%%%%%%Beamer Settings and Macros %%%%%%%%%%%%%%%
%Turn-off the stupid symbols at the bottom
\setbeamertemplate{navigation symbols}{}

%%% Font sizes
\setbeamerfont{itemize/enumerate body}{size=\normalsize}
\setbeamerfont{itemize/enumerate subbody}{size=\small}
\setbeamerfont{itemize/enumerate subsubbody}{size=\footnotesize}

%Iowa State Red and Gol
\definecolor{ISURed}{RGB}{200, 16, 46}
\definecolor{ISUGold}{RGB}{241, 190, 72}
\definecolor{ISUBrown}{RGB}{82,71,39}

%\setbeamercolor{structure}{fg=ISURed}
%\setbeamercolor{title}{fg=ISUBrown}
%\setbeamercolor{frametitle}{fg=ISUBrown}
%\setbeamercolor{palette primary}{use=structure,fg=ISUGold,bg=ISURed}
%\setbeamercolor{palette secondary}{fg=ISURed,bg=ISUGold}
%\setbeamercolor{palette tertiary}{fg=ISUGold,bg=ISURed} 
%\setbeamercolor{item projected}{fg=ISUGold}

%Ames Lab Blues
\definecolor{AmesBlue1}{RGB}{33, 71, 118}
\definecolor{AmesBlue2}{RGB}{197, 219, 240}
\definecolor{AmesBlue3}{RGB}{214, 224, 236}

\setbeamercolor{structure}{fg=AmesBlue1}
\setbeamercolor{title}{fg=AmesBlue2,bg=AmesBlue1}
\setbeamercolor{frametitle}{fg=AmesBlue2,bg=AmesBlue1}
\setbeamercolor{palette primary}{use=structure,fg=AmesBlue2,bg=AmesBlue1}
\setbeamercolor{palette secondary}{fg=AmesBlue1,bg=AmesBlue2}
\setbeamercolor{palette tertiary}{fg=AmesBlue2,bg=AmesBlue1} 
\setbeamercolor{item projected}{fg=AmesBlue2}

\hypersetup{colorlinks=true, urlcolor=AmesBlue1, linkcolor=AmesBlue1}
%Switch colors 


%Make points you haven't shown invisible, and the ones you have grayed out
\setbeamercovered{invisible}
\setbeamercovered{again covered={\opaqueness<1->{50}}}

%Load the settings for making the citation footnotes
\setbeamerfont{footnote}{size=\tiny}
\usepackage[numbers]{natbib}
\usepackage{bibentry}
\bibliographystyle{abbrv}

%Makes the title slide
\newcommand{\TitleSlide}{
\begin{frame}
\titlepage
\end{frame}
}

\newcommand{\BulletList}[1]{
	\begin{block}{}
		\begin{itemize}
			#1
		\end{itemize}
	\end{block}	
}

%A two-column slide, where arg1 is the title, arg2 is
%column one, arg3 is column two, and arg4 is a footer
\newcommand{\TwoColumnSlide}[4]{
\begin{frame}[squeeze]{#1}
\setlength{\leftmargini}{0.5cm}
\setlength{\leftmarginii}{0.5cm}
\begin{columns}[c,t, totalwidth=1.0\textwidth]
\begin{column}[c]{0.49\linewidth}
#2
\end{column}
\begin{column}[c]{0.49\linewidth}

#3

\end{column}
\end{columns}

#4

\end{frame}}

\newcommand{\TwoRowSlide}[4]{
\begin{frame}[squeeze]{#1}
\setlength{\leftmargini}{0.5cm}
\setlength{\leftmarginii}{0.5cm}
\begin{minipage}[c][0.49\textheight][c]{0.9\textwidth}
\begin{center}
#2
\end{center}
\end{minipage}

\begin{minipage}[c][0.49\textheight][c]{0.9\textwidth}
\begin{center}
#3
\end{center}
\end{minipage}
#4
\end{frame}}


%Adds a picture, with the path given as arg1, to a slide, respecting the sizes of the environment and not distorting the picture.  Also works with Beamer's overlays
\newcommand<>{\AddPicture}[1]{
\begin{center}
\includegraphics#2[width=0.90\textwidth,height=0.95\textheight,keepaspectratio]{#1}
\end{center}
} 

% Starts the backup slides
\newcommand{\BackupBegin}{
	\newcounter{finalframe}
	\setcounter{finalframe}{\value{framenumber}}
}

% Ends the backup slides
\newcommand{\BackupEnd}{
	\setcounter{framenumber}{\value{finalframe}}
}

%%%%%%%%%%%%%%%%%%% End Macros and Stuff %%%%%%%%%%%%%%

\AtBeginSection[]{
	\begin{frame}
		\vfill
		\centering
		\begin{beamercolorbox}[sep=8pt,center,shadow=true,rounded=true]{title}
			\usebeamerfont{title}\insertsectionhead\par%
		\end{beamercolorbox}
		\vfill
	\end{frame}
}

\definecolor{codegreen}{rgb}{0,0.6,0}
\definecolor{codegray}{rgb}{0.5,0.5,0.5}
\definecolor{codepurple}{rgb}{0.58,0,0.82}
\definecolor{backcolour}{rgb}{0.95,0.95,0.92}

\lstdefinestyle{mystyle}{
    backgroundcolor=\color{backcolour},   
    commentstyle=\color{codegreen},
    keywordstyle=\color{magenta},
    numberstyle=\tiny\color{codegray},
    stringstyle=\color{codepurple},
    basicstyle=\ttfamily\tiny,
    breakatwhitespace=false,         
    breaklines=true,                 
    captionpos=b,                    
    keepspaces=true,                 
    numbers=left,                    
    numbersep=5pt,                  
    showspaces=false,                
    showstringspaces=false,
    showtabs=false,                  
    tabsize=2,
    xleftmargin=0.05\textwidth,
    xrightmargin=0.45\textwidth	
}

\newcommand{\inlinecode}[1]{\colorbox{backcolour}{\lstinline{#1}}}

\lstset{style=mystyle}

\author[RMR]{\textbf{Ryan~M.~Richard}}
\title[Python101]{Python Practice Session}
\institute[Ames]{
	Scientist II and Adjunct Assistant Professor of Chemistry\\
	Ames National Laboratory and Iowa State University\\
	Ames, IA, USA
} 

\date[5-30-2025]{2025 SIMCODES Bootcamp}

\begin{document}
\TitleSlide

\TwoColumnSlide{Objectives}
{\BulletList{
	\item Understand the basics of Python software.
	\item Write a Python package for the Fibonacci sequence.
}}
{}
{}

\defverbatim[colored]\listOne{
\begin{lstlisting}[language=python]
# This is a comment. 
# It gives the reader context.
print('hello world')
for i in [1, 2, 3]:
    print(i)
\end{lstlisting}
}

\TwoColumnSlide{Notes on Writing Python}
{\BulletList{
	\item Case-sensitive, e.g., \texttt{Print} is not the same as \texttt{print}
	\item Leading spaces are significant (and should always be spaces, \textbf{NOT} tabs).
	\begin{itemize}
		\item The number of spaces is up to you, but you should be consistent.
		\item e.g., if you choose to indent two spaces, you should always use two space.
	\end{itemize}
	\item The type of brackets, braces, etc. are significant.
}}
{
\listOne
}
{}

\TwoColumnSlide{What is a Notebook?}
{\BulletList{
	\item Examples are Colab and Jupyter.
	\item Notebooks are single files (\texttt{*.ipynb} extensions).
	\item Notebooks are designed for singe use.
	\begin{itemize}
		\item E.g., tutorials, data analysis, rapid prototyping.  
	\end{itemize}
	\item NOT intended for reuse, or further development.
	\begin{itemize}
		\item Can force it if you really want to...
	\end{itemize} 
}}
{
}

\TwoColumnSlide{What is a Module?}
{
\BulletList{
	\item Python software has three flavors: modules, packages, and libraries.
	\item A module is a single \texttt{*.py} file.
	\item Modules are building blocks of packages and libraries.
	\item All the actual effort goes into writing modules.
}}
{}
{}


\TwoColumnSlide{What is a Package?}
{\BulletList{
        \item Packages are modules distributed together.
	\item Directory, \texttt{\_\_init\_\_.py} file, and \texttt{*.py} files.
	\item Name of the directory will be the name of the package.
	\item The \texttt{\_\_init\_\_.py} file is usually empty, but is essential because
	         that's how Python determines your directory is a package.
	\item The \texttt{*.py} files are plain text and are where you put the code. 
}}
{
}
{}

\TwoColumnSlide{What is a Library?}
{\BulletList{
	\item Libraries are packages that are distributed together.
	\item Libraries are usually distributed individually so the
	        terms stop here.
	\item We won't worry about how to develop a library.
}}
{}
{}

\TwoColumnSlide{What is a project?}
{\BulletList{
	\item Disclaimer, may not be a standardized term.
	\item Modules, packages, and libraries are ``just the code.''
	\item Good software also ``infrastructure'', e.g., documentation, tests, etc.
	\item We call everything together the ``project.''
}}
{}
{}

\TwoColumnSlide{Fibonacci Sequence}
{\BulletList{
	\item 0, 1, 1, 2, 3, 5, 8, $\ldots$
	\item $F_i = F_{i-2} + F_{i-1}$
	\item Our goal: write a package for computing $F_i$.
}}
{}
{}

\TwoColumnSlide{Setup: Project Structure}
{\BulletList{
	\item<1-> Let's call our project \texttt{Fibby}
	\item<1-> In terminal, start by making a directory called \texttt{fibby}.
	\begin{itemize}
		\item<1-> Note it is lowercase to avoid file system gotchas.	
	\end{itemize}
	\item<1-> What is the terminal command for creating \texttt{fibby}?
	\begin{itemize}
		\item<2-> \inlinecode{mkdir fibby}
	\end{itemize}
	\item<3> All of our project files will live inside this directory.
	\item<3> Practice: add \texttt{fibby/README.md} with contents:
	        \inlinecode{Welcome to Fibby!}
	\item<3> N.b., paths for this tutorial are given relative to the first \texttt{fibby}
	               directory you made.
}}
{}
{}

\TwoColumnSlide{Setup: Package Structure}
{\BulletList{
	\item Within a project, Python modules are separated from infrastructure.
	\item Python modules for a package are stored in a single directory.
	\item Python uses the directory name as the name of the package.
	\item Unfortunately this leads to a very common pattern where the name of
	         the project and the name of the package are the same.
	\item I.e., add a directory \texttt{fibby/fibby}
}}
{}
{}

\TwoColumnSlide{Setup: Making \texttt{fibby} a Package}
{\BulletList{
	\item<1-> Python doesn't know \texttt{fibby/fibby} is a package.
	\item<1-> How do we tell Python that \texttt{fibby} is a package?
	\begin{itemize}
		\item<2> Add a file \texttt{\_\_init\_\_.py} to \texttt{fibby/fibby}.
	\end{itemize}
}}
{}
{}

\defverbatim[colored]\listTwo{
\begin{lstlisting}[language=python]
def fibonacci(i):
    if i == 0:
        return 0
    if i == 1:
        return 1
\end{lstlisting}
}

\TwoColumnSlide{Setup: Making the \texttt{fibby} Module}
{\BulletList{
	\item Create a file \texttt{fibby/fibby/fibonacci.py}.
	\item For now, just worry about computing $F_0=0$ and $F_1=1$.
	\item Write a function which takes $i$ and returns $F_i$.
}}
{
\only<2>{\listTwo}
}
{}

\TwoColumnSlide{Does Fibby work?}
{\BulletList{
	\item<1> What is the command to run Fibby?
	\begin{itemize}
		\item<2> \inlinecode{python3 fibonacci.py}
		\item<2> \inlinecode{python3 fibonacci.py 0}
		\item<2> \inlinecode{python3 fibonacci.py 1}
	\end{itemize}
	\item<3> This was a trick question. We only \textbf{defined} a function \texttt{fibonacci}.
	                We never called it.
	\item<3> Very common error...
}}
{}
{}


\defverbatim[colored]\listThree{
\begin{lstlisting}[language=python]
def fibonacci(i):
    if i == 0:
        return 0
    if i == 1:
        return 1
 

fibonacci(0)
\end{lstlisting}
}

\defverbatim[colored]\listFour{
\begin{lstlisting}[language=python]
def fibonacci(i):
    if i == 0:
        return 0
    if i == 1:
        return 1
 

print(fibonacci(0))
\end{lstlisting}
}

\TwoColumnSlide{Calling Fibby.}
{\BulletList{
	\item<1> Add a call to \texttt{fibonacci}.
	\item<1> Now how do we run it?
	\begin{itemize}
		\item<2> \inlinecode{python3 fibonacci.py}
	\end{itemize}
	\item<3> This was a trick question. We called the function, but didn't \textbf{print} the value.
	\item<3> Again common error. Remember computers do \textbf{exactly} what you ask them
	               to do.
	\item<4> Correct code.
	\item<4> How would we print $F_1$?
}}
{
\only<1-3>{\listThree}
\only<4>{\listFour}
}
{}

\defverbatim[colored]\listFour{
\begin{lstlisting}[language=python]
def fibonacci(i):
    if i == 0:
        return 0
    if i == 1:
        return 1
\end{lstlisting}
}


\TwoColumnSlide{Development Notes}
{\BulletList{
	\item When you develop software you typically don't call your function, class, etc. in the module
	         you define it in.
	\item We did this just as a ``sanity check.''
	\item Want to ``separate concerns.'' 
	\item (Go ahead and delete the \inlinecode{print(fibonacci(0))} line).
	\item Modern software development is test-driven. So lets write a test.          
}}
{
\listFour
}
{}

\defverbatim[colored]\listFive{
\begin{lstlisting}[language=python]
from fibby.fibonacci import fibonacci

assert fibonacci(0) == 0
assert fibonacci(1) == 1
\end{lstlisting}
}

\TwoColumnSlide{Testing Fibby}
{\BulletList{
	\item Create \texttt{fibby/tests}
	\item Create \texttt{fibby/tests/\_\_init\_\_.py}
	\item Create {\small \texttt{fibby/tests/test\_fibonacci.py}}
	\item Run: \inlinecode{pytest} in \texttt{fibby} directory.
	\item At this point
}}	
{
\listFive
}
{}

\defverbatim[colored]\listSix{
\begin{lstlisting}[language=python]
# Project Structure
# project_name/
# |--- package_name/
# |    |---__init__.py
# |    |---module1_name.py
# |    |---module2_name.py
# |--- tests/
# |    |--- __init__.py
# |    |--- test_module1_name.py
# |    |--- test_module2_name.py
\end{lstlisting}
}

\TwoColumnSlide{Notes on Structure}
{\BulletList{
	\item We now have a bare-bones Python project.
	\item There's actually theory behind why projects are set up this way.
	\item If you try to get creative you will likely run into import errors.
	\item Until you are a Python expert, just stick with this pattern.
}}
{
\listSix
}
{}

\defverbatim[colored]\listSeven{
\begin{lstlisting}[language=python]
def fibonacci(i):
    seq = [0, 1]
    
    while len(seq) <= i:
        seq.append(seq[-2] + seq[-1])
        
    return seq[i]
\end{lstlisting}
}

\TwoColumnSlide{Towards Arbitrary $F_i$}
{\BulletList{
	\item Our code only works for $i\le 1$.
	\item Let's generalize it.
	\item Many possible implementations.
	\item General tip: get something that works first.
	\item Easy implementation: fill a list.
}}
{
\only<2>{\listSeven}
}
{}

\TwoColumnSlide{Performance}
{\BulletList{
	\item<1-> Our implementation is not optimal.
	\item<1-> How bad do you think it is?
	\item<1-> Try asking for $F_{10}=55$
	\item<2-> $F_{54}=86267571272$
	\item<3-> $F_{502}=365014740723634211012237077$
	               $906479355996081581501455497$
	               $852747829366800199361550174$
	               $096573645929019489792751$
	\item<4-> $F_{5008}=182\cdots$
	\item<5-> $F_{25000}=???$ (string conversion error)
}}
{}
{}

\TwoColumnSlide{Performance: Lesson}
{\BulletList{
	\item Pro tip: do NOT pre-maturely optimize your code.
	\item Our naive Fibonacci code can compute $F_{25000}$ in under a second!
	\item Code readability/maintainability should be the priority for the first pass.
	\item Only start optimizing code when it becomes a bottleneck.
}}
{}
{}

\defverbatim[colored]\listSeven{
\begin{lstlisting}[language=python]
def fibonacci(i):
    '''
    This is a summary line. It should be 
    about a sentence long.
    
    This is the extended description. 
    Note that the entire docstring is 
    indented as far as the quotes we used.
    
    Parameters
    ----------
    i : int
       Which fibonacci number you want. 
       i == 0 is 0. i == 1 is 1. 
       i == 2 is 1. i == 3 is 2. Etc.
    
    Returns
    -------
    int
        The requested fibonacci number
    '''
    
    # Code would go here
\end{lstlisting}
}

\TwoColumnSlide{Documentation}
{\BulletList{
	\item I recommend adding documentation as the last step before a PR.
	\begin{itemize}
		\item If you add it too early, you'll often find that you need to change it because
		         you added/removed a parameter, changed the type of a parameter, etc.
	\end{itemize}
	\item In Python we document functions with ``docstrings''.
	\item Docstrings are free-form, but in science we usually follow ``numpydoc'' style.
	\begin{itemize}
		\item \url{https://www.geeksforgeeks.org/python-docstrings/}
	\end{itemize}
	
}}
{
\listSeven
}
{}


\TwoColumnSlide{Generalizing}
{\BulletList{
	\item While real-world problems are harder, they're tackled basically the same way.
         \item Write code units (e.g., function, class, libraries) for each task.
         \item Test the code units as you write them.
         \item If code unit gets too big, split it.
         \item Document before committing.
}}
{}
{}

\TwoColumnSlide{Misc.}
{\BulletList{
	\item Software engineering is ultimately a soft science.
	\item Should this be a function? A class? A library? Subjective.
	\item Ultimately, any solution that gets the right answer 
	         (for the right reasons) is ``correct''.
	\item Easier to test small code units. I recommend erring on the
	         side of too small, rather than too large.
	\item A lot like ``showing your work'' in math. Each code unit shows
	         a step. 
	\item As you get more comfortable you can start ``skipping steps.''
}}
{}
{}

\TwoColumnSlide{Debugging}
{\BulletList{
	\item In practice, most of coding is debugging.
	\item Something you get better at with practice.
	\item Need to trace the logic of the program and see where if fails.
	\item Usually introduced through \inlinecode{print} statements.
	\begin{itemize}
		\item Basically print values until you find the first one that
		         is wrong.
	\end{itemize}
	\item Tools called ``debuggers'' that can do this (and more) for
	         you.
}}
{}
{}

\TwoColumnSlide{Practice}
{\BulletList{
	\item Fibonacci sequence prototypical example of recursion.
	\begin{itemize}
		\item Recursion: Function \texttt{fibonacci} calls itself.
	\end{itemize}
	\item Goal: write the call for $F_i$ in terms of the calls for
	         $F_{i-2}$ and $F_{i-1}$.
	\item Hint: in terminal \texttt{ctrl + c} kills the current running
	         process.
	\begin{itemize}
		\item Common recursion problem is running forever because
		         function just keeps calling itself.
	\end{itemize}
}}
{}
{}

\TwoColumnSlide{Acknowledgements}
{\BulletList{
	\item NSF for funding the REU.
	\item ISU and Ames National Lab
}}
{
	\AddPicture{nsf.jpg}
}
{}

\end{document}
